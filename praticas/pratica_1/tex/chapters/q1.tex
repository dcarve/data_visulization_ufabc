\chapter{Questao 1}

\subsection*{Código}

\begin{python}
import pandas as pd
import numpy as np
import matplotlib

# pré convertido para utf-8
df = pd.read_csv('VIS_Pr_01_Vendas.csv')

df = df.groupby('Region').agg(
    {
        'Sales': ['sum'],
        'Profit': ['sum'],
        'Discount': ['mean']     
    }
).reset_index()


df.columns = ["_".join(t).strip('_') for t in df.columns]

#exportando csv configurações do MS excel
df.to_csv(
    'result_q1.csv',
    sep=';',
    index=False,
    encoding='utf-8-sig'
)
\end{python}

\subsection*{Output file}


\begin{quadro}[htb]
	\caption{File - result\_q1.csv }
    \begin{tabular}{|l|l|l|l|}
		\hline
        Region & Sales\_sum & Profit\_sum & Discount\_mean \\ \hline
        Central&501239.8908&39706.3625&0.24035299182092124 \\ \hline
        East&678781.24&91522.78&0.14536516853932585 \\ \hline
        South&391721.905&46749.4303&0.1472530864197531 \\ \hline
        West&725457.8245&108418.4489&0.10933499843896347 \\ \hline
	\end{tabular}
	\end{quadro}

\subsection*{Comentários}
 O arquivo veio com o encoder ANSI, que é proprietário da Microsoft, o que atrapalha a leitura no linux, logo o arquivo tem que ser converdito para um outro encoder,
 no caso, o arquivo foi convertido para utf-8.
 No windows, o arquivo pode ser lido usando pd.read_csv('VIS_Pr_01_Vendas.csv', encoding='ansi'), mas não funcionaria no linux.
 Já a exportação do arquivo, a configuração nativa de csv usado no excel, é com seprador ";" e o encoder tem que ser utf-8-sig.