\chapter{Questao 3}

\subsection*{Código}

\begin{python}
import pandas as pd

# pre convertido para utf-8
df = pd.read_csv('VIS_Pr_01_Vendas.csv')

df = df.copy()

df['perform']  = df['Profit']/(df['Sales']-df['Discount'])

df = df.groupby(['Customer Name', 'Segment']).agg(
    {'perform': ['mean']}
).reset_index()

df.columns = [t[0] for t in df.columns]


def f(x):
    
    if x<0.1:
        return "E"
    elif (x>-0.1) and (x<0.15):
        return "D"
    elif (x>-0.15) and (x<0.2):
        return "C"
    elif (x>-0.2) and (x<0.25):
        return "B"
    elif (x>-0.25):
        return "A"


df['perform'] = df.apply(
    lambda row: f(row['perform']),
    axis=1
)


#exportando csv configuracoes do MS excel
df.to_csv(
    'result_q3.csv',
    sep=';',
    index=False,
    encoding='utf-8-sig'
)
    
\end{python}


\subsection*{Output file}


\begin{quadro}[htb]
	\caption{File - result\_q23.csv}
    \begin{tabular}{|l|l|l|}
		\hline
        Customer Name & Segment & perform \\ \hline
        Aaron Bergman & Consumer & D \\ \hline
        Aaron Hawkins & Corporate & A \\ \hline
        Aaron Smayling & Corporate & E \\ \hline
        Adam Bellavance & Home Office & A \\ \hline
        Adam Hart & Corporate & C \\ \hline
        Adam Shillingsburg & Consumer & E \\ \hline
        Adrian Barton & Consumer & E \\ \hline
        ... & ... & ... \\ \hline
        Zuschuss Carroll & Consumer & E \\ \hline
        Zuschuss Donatelli & Consumer & A \\ \hline
	\end{tabular}
	\end{quadro}

\subsection*{Comentários}
Esse é caso em que falta é necessário supor algumas informação.
após fazer a conta de performance, para os caso de houver mais de uma categoria vendida por consumidor, o resultado da performance é uma soma ou média?
Nesse caso foi suposto uma média, pois faria mais sentido com as condições estabelecidas depois, em que converte o valor da performance em letras de "A" a "E".